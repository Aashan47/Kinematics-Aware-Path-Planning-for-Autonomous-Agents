\documentclass[11pt,a4paper]{article}
\usepackage{amsmath,amssymb,amsfonts}
\usepackage{graphicx}
\usepackage{geometry}
\geometry{margin=2.5cm}

\title{Kinematics-Aware Autonomous Navigation:\\Mathematical Foundations for Differential Drive Robots}
\author{MSc AI \& Robotics Application - Sapienza University of Rome}
\date{\today}

\begin{document}
\maketitle

\section{State-Space Representation}

The configuration of a differential drive robot in the 2D plane is described by the state vector:
\begin{equation}
\mathbf{q} = \begin{bmatrix} x \\ y \\ \theta \end{bmatrix} \in SE(2)
\end{equation}

where:
\begin{itemize}
    \item $x, y \in \mathbb{R}$ represent the position of the robot's center (midpoint between wheels) in the world frame
    \item $\theta \in (-\pi, \pi]$ is the orientation angle measured from the positive x-axis
\end{itemize}

The robot operates under \textbf{nonholonomic constraints}, meaning it cannot move instantaneously in the lateral direction. This constraint is expressed as:
\begin{equation}
\dot{x}\sin\theta - \dot{y}\cos\theta = 0
\end{equation}

\section{Forward Kinematics}

\subsection{From Wheel Velocities to Body Velocities}

Given the angular velocities of the left ($\omega_L$) and right ($\omega_R$) wheels, and wheel parameters:
\begin{itemize}
    \item $r$ = wheel radius [m]
    \item $L$ = wheelbase (distance between wheel centers) [m]
\end{itemize}

The linear and angular velocities of the robot body are:

\begin{equation}
\boxed{v = \frac{r}{2}(\omega_R + \omega_L)}
\end{equation}

\begin{equation}
\boxed{\omega = \frac{r}{L}(\omega_R - \omega_L)}
\end{equation}

In matrix form:
\begin{equation}
\begin{bmatrix} v \\ \omega \end{bmatrix} = \frac{r}{2} \begin{bmatrix} 1 & 1 \\ \frac{2}{L} & -\frac{2}{L} \end{bmatrix} \begin{bmatrix} \omega_R \\ \omega_L \end{bmatrix}
\end{equation}

\subsection{Kinematic Model (State Evolution)}

The time evolution of the robot state follows the \textbf{unicycle model}:

\begin{equation}
\boxed{\dot{\mathbf{q}} = \begin{bmatrix} \dot{x} \\ \dot{y} \\ \dot{\theta} \end{bmatrix} = \begin{bmatrix} \cos\theta & 0 \\ \sin\theta & 0 \\ 0 & 1 \end{bmatrix} \begin{bmatrix} v \\ \omega \end{bmatrix}}
\end{equation}

Expanded form:
\begin{align}
\dot{x} &= v \cos\theta \\
\dot{y} &= v \sin\theta \\
\dot{\theta} &= \omega
\end{align}

\subsection{Discrete-Time Integration (Euler Method)}

For implementation with sampling period $\Delta t$:

\begin{equation}
\begin{bmatrix} x_{k+1} \\ y_{k+1} \\ \theta_{k+1} \end{bmatrix} = \begin{bmatrix} x_k \\ y_k \\ \theta_k \end{bmatrix} + \begin{bmatrix} v_k \cos\theta_k \\ v_k \sin\theta_k \\ \omega_k \end{bmatrix} \Delta t
\end{equation}

For improved accuracy (exact integration assuming constant $v$, $\omega$ over $\Delta t$):

\begin{equation}
\begin{cases}
x_{k+1} = x_k + \frac{v_k}{\omega_k}\left(\sin(\theta_k + \omega_k \Delta t) - \sin\theta_k\right) & \text{if } \omega_k \neq 0 \\
y_{k+1} = y_k + \frac{v_k}{\omega_k}\left(\cos\theta_k - \cos(\theta_k + \omega_k \Delta t)\right) & \text{if } \omega_k \neq 0 \\
\theta_{k+1} = \theta_k + \omega_k \Delta t
\end{cases}
\end{equation}

\section{Inverse Kinematics}

\subsection{From Body Velocities to Wheel Velocities}

Given desired linear velocity $v$ and angular velocity $\omega$:

\begin{equation}
\boxed{\omega_R = \frac{1}{r}\left(v + \frac{L}{2}\omega\right)}
\end{equation}

\begin{equation}
\boxed{\omega_L = \frac{1}{r}\left(v - \frac{L}{2}\omega\right)}
\end{equation}

In matrix form:
\begin{equation}
\begin{bmatrix} \omega_R \\ \omega_L \end{bmatrix} = \frac{1}{r} \begin{bmatrix} 1 & \frac{L}{2} \\ 1 & -\frac{L}{2} \end{bmatrix} \begin{bmatrix} v \\ \omega \end{bmatrix}
\end{equation}

\section{Control-Oriented Formulation}

\subsection{Linear Transfer Function Perspective}

For small perturbations around a nominal trajectory, the system can be linearized. Consider the error dynamics for point-to-point navigation:

Let the position error be $e = \sqrt{(x_g - x)^2 + (y_g - y)^2}$ and heading error $\alpha = \text{atan2}(y_g - y, x_g - x) - \theta$.

The linearized dynamics in Laplace domain for the heading control loop:
\begin{equation}
\frac{\Theta(s)}{W(s)} = \frac{1}{s}
\end{equation}

where $\Theta(s) = \mathcal{L}\{\theta(t)\}$ and $W(s) = \mathcal{L}\{\omega(t)\}$.

For position control (assuming aligned heading):
\begin{equation}
\frac{X(s)}{V(s)} = \frac{\cos\theta_0}{s}
\end{equation}

\subsection{PID Controller Design}

For the heading control loop with transfer function $G(s) = \frac{1}{s}$, a PD controller suffices:
\begin{equation}
C(s) = K_p + K_d s = K_p \frac{1 + T_d s}{1}
\end{equation}

The closed-loop transfer function:
\begin{equation}
T(s) = \frac{C(s)G(s)}{1 + C(s)G(s)} = \frac{K_p + K_d s}{s + K_p + K_d s} = \frac{K_d s + K_p}{(1+K_d)s + K_p}
\end{equation}

For asymptotic stability: $K_p > 0$, $K_d \geq 0$.

\subsection{Complete PID Control Law}

The discrete PID controller output:
\begin{equation}
u[k] = K_p e[k] + K_i \sum_{j=0}^{k} e[j] \Delta t + K_d \frac{e[k] - e[k-1]}{\Delta t}
\end{equation}

With anti-windup (clamping):
\begin{equation}
I[k] = \text{clamp}\left(I[k-1] + e[k] \Delta t, -I_{\max}, I_{\max}\right)
\end{equation}

\section{3D Rigid Body Extension}

For a ground robot on uneven terrain, the full 6-DOF pose is:
\begin{equation}
\mathbf{T} \in SE(3) = \begin{bmatrix} \mathbf{R} & \mathbf{p} \\ \mathbf{0}^T & 1 \end{bmatrix}
\end{equation}

where $\mathbf{R} \in SO(3)$ is the rotation matrix (roll, pitch, yaw) and $\mathbf{p} = [x, y, z]^T$.

The angular velocity in body frame relates to Euler angle rates via:
\begin{equation}
\begin{bmatrix} \omega_x \\ \omega_y \\ \omega_z \end{bmatrix} = \begin{bmatrix} 1 & 0 & -\sin\phi \\ 0 & \cos\phi & \sin\phi\cos\theta \\ 0 & -\sin\phi & \cos\phi\cos\theta \end{bmatrix} \begin{bmatrix} \dot{\phi} \\ \dot{\theta} \\ \dot{\psi} \end{bmatrix}
\end{equation}

\end{document}
